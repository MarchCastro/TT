El presente proyecto contaba inicialmente con seis objetivos específicos derivados del objetivo general.  A partir de los módulos y prototipos planteados inicialmente, se obtienen los siguientes resultados:

\begin{itemize}
\item Se desarrolló el prototipo de una aplicación móvil dirigida a los clientes de tiendas departamentales misma que obtiene satisfactoriamente promociones, recomendaciones personalizadas, anuncios y notificaciones sobre productos de interés para el usuario, provenientes del prototipo de la aplicación móvil para vendedores. 
\item El módulo encargado de la generación de registros artificiales, funcionó correctamente ya que proporcionó datos ficticios para simular un entorno real.
\item El prototipo de la aplicación móvil dirigida al uso de los vendedores igualmente cumple las características planteadas inicialmente pues ofrece la funcionalidad de recomendar productos a los diferentes clientes y recibe notificaciones cuando un cliente requiere asistencia en sus compras.
\item Se implementó una plataforma en línea con el fin de que los administradores tengan la posibilidad de gestionar anuncios, mismas que visualiza el cliente en su aplicación móvil. Así mismo, tiene la opción de observar gráficamente el mercado que se tiene actualmente junto con sus características más importantes.
\item Los prototipos de ambas aplicaciones móviles, implementan la SDK de Estimote mediante la cual se obtiene la ubicación de los usuarios y de esta forma, el vendedor es capaz tanto de acudir con los clientes en caso de que requieran de ayuda como de realizar el envío de recomendaciones de productos.
\item Se puso en funcionamiento un servidor REST encargado de realizar operaciones de lectura y escritura sobre el repositorio de datos y capaz de proveer recomendaciones a las aplicaciones móviles mencionadas anteriormente.
\end{itemize}
%%%%% prototipo número cuatro
Con respecto al objetivo general en el Sistema de gestión, procesamiento y proveedor de datos de Retail se visualiza el diseño, implementación y resultados del sistema de recomendaciones basado en aprendizaje máquina, mismo que fungió como el núcleo de los módulos descritos anteriormente.

A partir de lo mencionado previamente, se puede corroborar que el proyecto propuesto proporcionó satisfactoriamente una solución a la problemática planteada al inicio del proyecto.
