%--------------------------------------------------
\section{Análisis de riesgos}
En los cuadros mostrados en la parte inferior se enlistan todos los riesgos potenciales presentes dentro del proyecto que podrían retrasar el tiempo de entrega o afectar directamente la elaboración de alguno de los prototipos planteados. Entre los riegos que se presentan, se consideran riesgos de tipo técnico, de negocio y del proyecto y de igual manera se enlistan las probabilidad de que estos sucedan utilizando por convención los siguientes valores \cite{Riesgos}: \\

\FloatBarrier
\begin{itemize}
\item Muy Bajo (\< 10\%)
\item Bajo ($<10\% - >25\%$)
\item Medio ($\<25\% - \>50\%$)
\item Alto ($\<50\% - \>75\%$)
\item Muy Alto ($\>75\%$) 
\end{itemize}
\FloatBarrier
En el cuadro \ref{table:riesgosnegocio} y \ref{table:riesgosnegocio2} se muestran los riesgos de negocio, estos se refieren a los peligros que corre el proyecto con respecto al entorno tanto económico como social en el que se desarrollará.
\FloatBarrier
\begin{table}[htb]
\setlength\extrarowheight{2pt} % for a bit of visual "breathing space"
\begin{tabularx}{\textwidth}{|C|C|C|C|C|}
\hline
\textbf{Tipo de riesgo} & \textbf{Riesgo} & \textbf{Probabilidad}  & \textbf{Impacto}  & \textbf{Plan de acción}
\\ \hline
Riesgo de negocio & Competencia respecto a sistemas desarrollados y probados en tiendas departamentales. & 25\% & Bajo & Introducir el producto poco a poco en el mercado a pesar de que la respuesta del público no sea inmediata.
\\ \hline
Riesgo de negocio & Precio superior respecto a sistemas similares. & 10\% & Bajo & Divulgación de los elementos extras con los que cuenta el sistema.
\\ \hline
Riesgo de negocio & Precio de venta inferior al requerido para recompensar costos de desarrollo. & 15\% & Bajo & Realizar un estudio de mercado con el fin de comprobar si realmente está correcto el precio de venta de la competencia y el nuestro.
\\ \hline
Riesgo de negocio & Dificultad para la venta del proyecto. & 70.8\% & Alto & Promover los servicios que el sistema ofrece a las diferentes tiendas departamentales y plazas comerciales.
\\ \hline
Riesgo de negocio & Poco conocimiento por parte del público sobre los beneficios que los dispositivos Beacon proveen. & 29.2\% & Medio & Realizar campañas publicitarias acerca de las ventajas que el uso de los Beacons proporciona.
\\ \hline
Riesgo de negocio & Falta de conocimiento de estrategias de venta y promoción del sistema. & 60\% & Alto & Contratar o solicitar ayuda de especialistas de ventas.
\\ \hline
\end{tabularx}
\caption{Riesgos de negocio que presenta el proyecto. }
\label{table:riesgosnegocio}
\end{table}

\FloatBarrier
\begin{table}[htb]
\setlength\extrarowheight{2pt} % for a bit of visual "breathing space"
\begin{tabularx}{\textwidth}{|C|C|C|C|C|}
\hline
\textbf{Tipo de riesgo} & \textbf{Riesgo} & \textbf{Probabilidad}  & \textbf{Impacto}  & \textbf{Plan de acción}
\\ \hline
Riesgo de negocio & Poca disposición por parte de los clientes para hacer uso de su información pública de Facebook. & 46.2\% & Medio & Distribución de información para hacer del conocimiento de los usuarios la forma en la que sus datos serán utilizados.
\\ \hline
Riesgo de negocio & Poca disposición por parte de los clientes para compartir su historial de compras de una tienda departamental. & 40\% & Medio & Distribución de información para hacer del conocimiento de los usuarios la forma en la que sus datos serán utilizados.
\\ \hline
Riesgo de negocio & Permisos denegados por parte de los clientes para compartir su ubicación a los vendedores dentro de una tienda departamental. & 53.8\% & Alto & Dar a conocer el motivo por el cual su ubicación será utilizada únicamente dentro de la tienda.
\\ \hline
\end{tabularx}
\caption{Riesgos de negocio que presenta el proyecto.}
\label{table:riesgosnegocio2}
\end{table}
\FloatBarrier
Por otro lado, el cuadro \ref{table:riesgosdeproyecto} presenta los riesgos de proyecto, los cuales se refieren a los diferentes peligros que pueden encontrar los integrantes del equipo de análisis y desarrollo internamente.
\FloatBarrier
\begin{table}[htb]
\setlength\extrarowheight{2pt} % for a bit of visual "breathing space"
\begin{tabularx}{\textwidth}{|C|C|C|C|C|}
\hline
\textbf{Tipo de riesgo} & \textbf{Riesgo} & \textbf{Probabilidad}  & \textbf{Impacto}  & \textbf{Plan de acción}
\\ \hline
Riesgo de proyecto & Pérdida de información y documentación del sistema. & 5\% & Bajo & Realizar respaldos diarios de toda la información y actualizaciones hechas al proyecto y a la documentación de este.
\\ \hline
Riesgo de proyecto & Incumplimiento de los requerimientos establecidos.  & 30\% & Medio & Cumplir en tiempo y forma en la medida de lo posible con las actividades planificadas en el cronograma.
\\ \hline
Riesgo de proyecto & Falta de presupuesto para adquirir más dispositivos Beacon. & 10\% & Bajo & Contabilizar el número de Beacons requeridos para obtener el correcto desempeño de la aplicación y tener en cuenta un poco más del presupuesto que se enfoca a estos dispositivos.
\\ \hline
Riesgo de proyecto & No disponibilidad del personal requerido para el desarrollo del proyecto. & 30\% & Medio & Anticipar el trabajo en los tiempos libres con el fin de evitar un retraso en caso de que se presente una situación en la cual algún integrante no pueda presentarse para una reunión de trabajo.
\\ \hline
Riesgo de proyecto & Necesidad de escalabilidad del sistema. & 75\% & Muy Alto &  Contemplar el uso de software como bases de datos relacionales orientadas a objetos, que permitan una posterior escalabilidad del sistema.
\\ \hline
\end{tabularx}
\caption{Riesgos de proyecto que presenta el proyecto. }
\label{table:riesgosdeproyecto}
\end{table}
\FloatBarrier
Finalmente, el cuadro \ref{table:riesgostecnicos} muestra los riesgos técnicos, referidos a aquellos que se presentan en el equipo de hardware y/o software utilizados para la elaboración del sistema.
\FloatBarrier
\begin{table}[htb]
\setlength\extrarowheight{2pt} % for a bit of visual "breathing space"
\begin{tabularx}{\textwidth}{|C|C|C|C|C|}
\hline
\textbf{Tipo de riesgo} & \textbf{Riesgo} & \textbf{Probabilidad}  & \textbf{Impacto}  & \textbf{Plan de acción}
\\ \hline
Riesgo técnico & Adquisición de licencias de desarrollo no contempladas. & 10\% & Muy bajo & Contemplar el presupuesto requerido para la adquisición de licencias que se llegaran a necesitar.
\\ \hline
Riesgo técnico & Falla o baja batería de dispositivos Beacon. & 20\% & Bajo & Contemplar desde un inicio un porcentaje del presupuesto a fin de solicitar posteriormente más dispositivos. De igual manera, se deberá contemplar el tiempo que tarda el envío de dichos componentes.
\\ \hline
Riesgo técnico & Pérdida parcial o total de la información personal de los usuarios. & 15\% & Bajo & Respaldar cada 3 días los datos e información recopilada durante las jornadas.
\\ \hline
Riesgo técnico & Falla o pérdida de alguna de las computadoras contempladas para el desarrollo del sistema. & 50\% & Medio & Respaldar continuamente la información almacenada en cada una de los equipos de cómputo. Así mismo, conocer los tiempos de disponibilidad de los laboratorios de la escuela en caso de requerir su utilización.
\\ \hline 
\end{tabularx}
\caption{Riesgos técnicos que presenta el proyecto.}
\label{table:riesgostecnicos}
\end{table}
\FloatBarrier
