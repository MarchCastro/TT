
Las siguientes son algunas de las limitaciones que presenta el cómputo móvil, las cuales son descritas en el artículo “Mobile Computing: Principles, Devices and Operating Systems” \cite{MBPDO}. \\

\begin{itemize}
\item \textbf{Estándares de seguridad}: Cuando se trabaja en dispositivos móviles, en ocasiones se depende de las redes públicas, que requieren un uso cuidadoso de VPN. La seguridad es una preocupación importante al respecto de los estándares de computación móvil. Se podría atacar fácilmente la VPN a través de una gran cantidad de redes interconectadas a través de la línea \cite{MBPDO}.

\item \textbf{Seguridad en teléfonos inteligentes}: Son los objetivos preferidos de los ataques. Estos ataques explotan las debilidades relacionadas con los teléfonos inteligentes que pueden provenir de medios de telecomunicación inalámbrica como redes Wi-Fi(Wireless Fidelity) y Sistema Global para comunicaciones Móviles (Global System for Mobile communications, por sus siglas en inglés GSM). También hay ataques que explotan las vulnerabilidades del software tanto del navegador Web como del sistema operativo. Finalmente, existen formas de software malicioso que se basan en el conocimiento débil de los usuarios promedio \cite{MBPDO}.

\item \textbf{Consumo de energía}: Cuando una toma de corriente o un generador portátil no está disponible, las computadoras móviles deben depender completamente de la energía de la batería. En combinación con el tamaño compacto de muchos dispositivos móviles, esto a menudo significa que se deben usar baterías inusualmente caras para obtener la duración de la batería necesaria \cite{MBPDO}.

\item \textbf{Interferencias de transmisión}: El clima, el terreno y el rango desde el punto de señal más cercano pueden interferir con la recepción de la señal. La recepción en túneles, algunos edificios y áreas rurales a menudo es pobre \cite{MBPDO}.

\item \textbf{Interfaz humana con el dispositivo}: Las pantallas y los teclados tienden a ser pequeños, lo que puede dificultar su uso. Los métodos de entrada alternativos, como reconocimiento de voz o escritura a mano, requieren capacitación \cite{MBPDO}.

\end{itemize}