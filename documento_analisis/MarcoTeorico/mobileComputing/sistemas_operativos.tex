Un sistema operativo móvil o Mobile OS es un conjunto de programas de bajo nivel que permite la abstracción de las peculiaridades del hardware específico del teléfono móvil y provee servicios a las aplicaciones móviles, que se ejecutan sobre él. Al igual que los PC que utilizan Windows, Linux o Mac OS, los dispositivos móviles tienen sus sistemas operativos, como hemos hecho mención anteriormente, tanto para los teléfonos inteligentes como las tabletas los sistemas operativos más utilizados son Android, iOS, Windows Phone, BlackBerry OS, entre otros. Los sistemas operativos móviles son mucho más simples y están más orientados a la conectividad inalámbrica, los formatos multimedia para móviles y las diferentes maneras de introducir información en ellos  \cite{MBOS}. \\

\subsubsection{Symbian}

Symbian era un sistema operativo móvil diseñado para teléfonos inteligentes desarrollado originalmente por Symbian Ltd. pero actualmente mantenido por Accenture. La plataforma Symbian es la sucesora de Symbian OS y Nokia Series 60.  La última versión, Symbian 3, se lanzó oficialmente en el cuarto trimestre de 2010 y se utilizó por primera vez en el Nokia N8. El primer teléfono Symbian, el teléfono inteligente con pantalla táctil Ericsson R380 fue lanzado en 2000 y fue el primero dispositivo que se comercializará como un “teléfono inteligente” \cite{MBPDO}. \\

En 2010, Nokia lanzó el teléfono inteligente Nokia N8 con una pantalla táctil capacitiva sin aguja, el primer dispositivo en usar el Symbian 3. Algunas estimaciones indican que la cantidad de dispositivos móviles con el sistema operativo Symbian hasta el final del segundo trimestre de 2010 fue de 385 millones. Symbian fue la plataforma de teléfonos inteligentes número uno por cuota de mercado desde 1996 hasta 2011 cuando 
cayó al segundo lugar detrás del sistema operativo Android de Google \cite{MBPDO}. \\

En febrero de 2011, Nokia anunció que reemplazaría a Symbian con Windows Phone como sistema operativo en todos sus futuros teléfonos inteligentes. Esta transición se completó en octubre de 2011, cuando Nokia anunció su primera línea de Teléfonos inteligentes Windows Phone 7.5, Nokia Lumia 710 y Nokia Lumia 800 \cite{MBPDO}. \\

\subsubsection{Windows}

Windows Phone (abreviado WP) es un sistema operativo móvil desarrollado por Microsoft, como sucesor de Windows Mobile. A diferencia de su predecesor fue enfocado en el mercado de consumo en lugar del mercado empresarial. Con Windows Phone, Microsoft ofreció una nueva interfaz de usuario que integró varios de sus servicios activos. Compitió directamente contra Android de Google e iOS de Apple. Su última versión fue Windows Phone 8.1, lanzado el 14 de abril de 2014 \cite{MBWINP}. \\

Debido a la evidente fragmentación de sus sistemas operativos, Microsoft anunció en enero de 2015 que daba de baja a Windows Phone, para enfocarse en un único sistema más versátil denominado Windows 10 Mobile, disponible para todo tipo de plataformas (teléfonos inteligentes, tabletas y computadoras). Sin embargo el 8 de octubre de 2017, el ejecutivo de Microsoft, Joe Belfiore, reveló que la compañía ya no desarrollaría nuevas funciones o hardware para teléfonos con Windows, debido a su baja participación en el mercado y la consiguiente falta de software de terceros para la plataforma. Microsoft abandonó en gran parte su negocio móvil, despidiendo a la mayoría de los empleados de Microsoft Mobile en 2016, en su lugar se centró en proporcionar aplicaciones y servicios compatibles con Android e iOS \cite{MBWIN10}. \\

\subsubsection{BlackBerry}

Blackberry OS, es un sistema operativo incluido teléfonos móviles de la compañía Canadiense Research In Motion (RIM, viene incorporado en los móviles que también llevan nombre Blackberry, seguido por el modelo correspondiente. Su nivel de seguridad es lo que hace a estos de teléfonos móviles los preferidos por los profesionales y empresarios, así como la privacidad y salvaguarda de sus datos privados, aunque en los últimos años han ganado lugar entre el público masivo, dada la incorporación de herramientas de entretenimiento exclusivas \cite{MBBLACK}. \\

La versión más actual del sistema operativo Blackberry OS, es la versión 7.0, aunque no es apta para todos sus terminales, solo los modelos más avanzados tienen soporte y actualización automática a esta última versión. Además, el sistema operativo Blackberry cuenta con un completo sistema de entretenimiento en el que se incluyen Blackberry Music y Blackberry Messenger dos aplicaciones esenciales para los usuarios de este sistema operativo \cite{MBBLACK}.\\


\subsubsection{iOS}

iOS (previamente iPhone OS) es un sistema operativo móvil desarrollado y distribuido por Apple Inc. Originalmente lanzado en 2007 para iPhone y iPod Touch, se ha ampliado para admitir otros dispositivos de Apple como el iPad y Apple TV. A diferencia de Windows CE de Windows (Windows Phone) y Android de Google, Apple no licencia iOS para instalación en hardware que no sea de Apple. A partir de 2012, la App Store de Apple contenía más de 700,000 aplicaciones, que colectivamente se han descargado más de 30 mil millones de veces \cite{MBPDO}.\\

La interfaz de usuario de iOS se basa en el concepto de manipulación directa, utilizando gestos multitáctiles. El control de la interfaz tiene elementos que consisten de pantallas deslizables, interruptores y botones. La respuesta a la entrada del usuario es inmediata y proporciona una interfaz fluida. La interacción con el OS incluye gestos como deslizar, tocar, pellizcar y pellizcar hacia atrás, todos los cuales tienen definiciones específicas dentro del contexto del sistema operativo iOS y su interfaz multitáctil. Acelerómetros internos son usados por algunas aplicaciones para responder al movimiento del dispositivo o rotándolo en tres dimensiones (un resultado común es cambiar de modo vertical a horizontal). iOS se deriva de OS X, con el que comparte la fundación Darwin y, por lo tanto, es un sistema operativo Unix. iOS es la versión móvil de Apple del sistema operativo OS X utilizado en las computadoras Apple. En iOS, hay cuatro capas de abstracción: la capa Core OS, la capa Core Services, la capa Media y la capa Cocoa Touch \cite{MBPDO}.\\


\subsubsection{Android}

Android es un sistema operativo basado en Linux diseñado principalmente para dispositivos móviles con pantalla táctil como teléfonos inteligentes y tabletas, desarrolladas por Google junto con Open Handset Alliance. Desarrollado inicialmente por Android Inc, a quien Google respaldó financieramente y luego compró en 2005. Android se dio a conocer en 2007 junto con la fundación de Open Handset Alliance, un consorcio de 86 empresas de hardware, software y telecomunicaciones dedicadas al avance de estándares abiertos para dispositivos móviles \cite{MBPDO}.\\
 
Google publicó el código de Android como código abierto, bajo la Licencia Apache. El proyecto de código abierto de Android (AOSP), liderado por Google, tiene la tarea de mantener y desarrollar aún más para Android. Adicionalmente, Android tiene una gran comunidad de desarrolladores que escriben aplicaciones (``aplicaciones'') que extienden la funcionalidad de los dispositivos. Los desarrolladores escriben principalmente en una versión personalizada de Java y las aplicaciones se pueden descargar de tiendas en línea tales como Google Play (anteriormente Android Market), la tienda de aplicaciones dirigida por Google o sitios de terceros. En junio de 2012, hubo más de 600,000 aplicaciones disponibles para Android y la cantidad estimada de aplicaciones descargadas de Google Play fue de 20 mil millones \cite{MBPDO}. \\

El primer teléfono con Android se vendió en octubre de 2008 y para finales de 2010, Android se había convertido en la plataforma de teléfonos inteligentes líder en el mundo. Tenía una cuota de mercado mundial de teléfonos inteligentes del 59\% al comienzo de 2012. Para mayo de 2017 Android cuenta con el 85\% del mercado mundial de teléfonos inteligentes, siguiendole iOS con el 14.7\%, Windows Phone con el 0.1\% y el resto de sistemas operativos igualmente con 0.1\% \cite{MBAND}. \\
