Desde la década de 1990 se han creado muchos tipos de computadoras portátiles, entre los más conocidos se encuentran listados a continuación: el asistente personal, teléfono inteligente, tableta, PC ultra móvil y computadoras corporales \cite{MBPDO}. \\

\subsubsection{Asistente digital personal (PDA) }

Un asistente digital personal (Personal Digital Assistant, por sus siglas en inglés PDA), también conocido como computadora de bolsillo o asistente de datos personales, es un dispositivo móvil que funciona como un administrador de información personal. Los PDA se consideran, en gran medida, obsoletos con el uso generalizado de los teléfonos inteligentes \cite{MBPDA}. \\

La mayoría de los PDA pueden acceder a Internet, intranets o extranets a través de métodos inalámbricos como Wi-Fi o inalámbrico. La mayoría de los PDA usan tecnología de pantalla táctil. El término PDA se utilizó por primera vez el 7 de enero de 1992 por el CEO (Chief Executive Officer) de Apple Computer, John Sculley, en el Consumer Electronics en en Las Vegas, Nevada, refiriéndose a Apple Newton \cite{MBPDA2}. \\


\subsubsection{Tableta}

Las tabletas son más grandes que un teléfono celular o un asistente digital personal. Son un tipo de dispositivos móviles integrados en una pantalla táctil plana y operados principalmente tocando la pantalla. No se coloca ningún teclado físico en ellos. A menudo utiliza un teclado virtual en pantalla, un lápiz óptico pasivo o un bolígrafo digital \cite{MBPDO}. \\
 
Los primeros ejemplos del concepto de tableta se originaron en los siglos XIX y XX principalmente como prototipos e ideas conceptuales. Los primeros dispositivos electrónicos portátiles comerciales basados en el concepto aparecieron a finales del siglo XX. Apple lanzó el iPad con sistema operativo y tecnología de pantalla táctil en 2010 y se convirtió en el primer éxito comercial en todo el mundo \cite{MBPDO}. \\

Esto provocó un nuevo mercado para la tableta y después de este éxito muchos otros fabricantes han producido versiones propias, incluyendo Samsung, HTC (High Tech Computer Corporation), Motorola, BlackBerry, Sony, Amazon, HP (Hewlett-Packard), Microsoft, Archos, etc. Entre tabletas, los principales sistemas operativos son iOS (Apple), Android (Google), Windows (Microsoft) y QNX(BlackBerry). En 2012, se informó que el 31\% de los usuarios de Internet en Estados Unidos tenían una tableta, que se utilizaba principalmente para visualizar contenido como videos y noticias  \cite{MBPDA3}. \\

\subsubsection{PC Ultra-Móvil}

Una computadora personal ultra móvil (Ultra Mobile PC, por sus siglas en inglés UMPC) es una versión de formato pequeño de una computadora de bolígrafo, una clase de computadora portátil cuyas especificaciones fueron lanzadas por Microsoft e Intel en la primavera de 2006. Sony con su serie Vaio U había fabricado el primer intento en esta dirección en 2004, que sin embargo solo se vendió en Asia. Los UMPCs son más pequeños que los subportátiles operados como tabletas, con una pantalla de Transistor de Películas Finas (Thin Film Transistor, por sus siglas en inglés TFT) que mide (en diagonal) alrededor de 12.7 a 17.8 cm, y una pantalla táctil o un lápiz óptico. No hay límites definidos entre los subportátiles y las PC ultra móviles \cite{MBPCUM}. \\

Los UMPC de primera generación eran simples computadoras personales (Personal Computer, por sus siglas en inglés PC) con Linux o una versión adaptada del sistema operativo de la tableta PC de Microsoft. Los UMPC de segunda generación usan menos electricidad y, por lo tanto, se pueden usar por más tiempo (hasta cinco horas) y también son compatibles con Windows Vista \cite{MBPCUM}. \\

Originalmente con el nombre en código Project Origami, el proyecto se lanzó en 2006 como una colaboración entre Microsoft, Intel, Samsung y algunos otros. A pesar de la predicción de la desaparición de la categoría de dispositivo UMPC según el sitio Web de multimedia estadounidenses ``CNET'', la categoría UMPC parece seguir existiendo, sin embargo, ha sido suplantada en gran medida por las tabletas \cite{MBPDO}. \\

\subsubsection{Computadoras corporales}

Las computadoras portátiles, también conocidas como computadoras corporales, son dispositivos electrónicos en miniatura que lleva el portador debajo, con o encima de la ropa . Una de las principales características de una computadora portátil es la consistencia. Hay una interacción constante entre la computadora y el usuario, es decir, no hay necesidad de encender o apagar el dispositivo. Otra característica es la capacidad de multitarea. No es necesario que el usuario deje de hacer lo que está haciendo para usar el dispositivo. Estos dispositivos pueden ser incorporados por el usuario para actuar como una prótesis \cite{MBPDO}.  \\

A comienzos del siglo XXI, la ``computadora corporal'' es un tema de investigación, cuyas áreas de estudio incluyen diseño de interfaces de usuario, realidad virtual y reconocimiento de patrones. El uso de estos dispositivos para aplicaciones específicas o para compensar discapacidades, así como apoyar en la estabilidad de los ancianos va en aumento. La aplicación de las computadoras corporales en el diseño de moda queda ejemplificada en el prototipo de Microsoft del "Vestido Impreso", presentado en el Simposio Internacional de Dispositivos Corporales en junio de 2011 \cite{MBCC}. \\

\subsubsection{Teléfono inteligente}

Un teléfono inteligente es un teléfono móvil construido en un sistema operativo móvil, con capacidad informática más avanzada y mejor conectividad que un simple teléfono \cite{MBPDO}. \\
 
Los primeros teléfonos inteligentes eran una combinación de un asistente digital personal (PDA) y las funcionalidades de un teléfono móvil. Algunas funcionalidades se agregaron en modelos posteriores, como reproducir multimedia, cámaras digitales compactas de gama baja, unidades de navegación GPS para formar un dispositivo multiuso, pantallas táctiles de alta resolución y navegadores Web para visualización de sitios Web estándar y páginas optimizadas para dispositivos móviles. Además, la conección por medio de Wi-Fi proporciona acceso a datos de alta velocidad y banda ancha móvil \cite{MBPDO}. \\

Los sistemas operativos móviles (Mobile Operative System, por sus siglas en inglés Mobile OS) más comunes utilizados por los teléfonos inteligentes modernos incluyen Android de Google, iOS de Apple, Symbian de Nokia, BlackBerry OS de RIM, Bada de Samsung y Windows Phone de Microsoft. Tales sistemas operativos son capaces de ajustarse con muchos modelos diferentes de teléfonos \cite{MBPDO}.  \\

