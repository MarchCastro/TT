
En aproximadamente 45 años desde la primer llamada a través de un teléfono inalámbrico, los teléfonos inteligentes mejor conocidos por su nombre en inglés, smartphones, han cambiado desde su tamaño y aspecto, hasta en sus funcionalidades. El diario periodístico ``Gestión de Perú" nos presenta una breve historia y evolución de los smartphones hasta el año 2000 la cual se muestra a continuación  \cite{MBHistoria}: \\ \par

\textbf{13 de abril de 1973: La primera llamada.}
Se realizó aquel día a través de un teléfono inalámbrico, cuando el directivo de Motorola Martin Cooper se comunicó desde una calle de Nueva York con su mayor rival en el sector, Joel Engel de Bell Labs de AT\&T.
\\ \par
\textbf{1 de agosto de 1994: IBM Simon Personal Communicator.}
Creado por IBM, fue el primer celular reconocido como Smartphone. Incluía características de un asistente personal de datos. Simon, fue el primer Smartphone con pantalla táctil. En 1995 se dejó de vender.
\\ \par
\textbf{1996: Nokia 9000 communicator.}
Mezclaba las capacidades de una agenda digital con las funciones de comunicación de un celular. Podían enviar correos, contaban con agenda para citas, capacidad para grabar contactos y hasta podían mandar documentos por fax.
\\ \par
\textbf{1997: Ericsson GS88.}
Ericsson describió su modelo GS88 “Penelope” como un “teléfono inteligente”, separando las dos palabras en inglés, “Smartphone”.
\\ \par
\textbf{1999: Benefon Esc.}
El segundo fabricante finlandés de teléfonos móviles, después de Nokia, lanzó el Benefon Esc, el primer teléfono con grabadora de mensajes y un receptor satelital tipo GPS.
\\ \par
\textbf{2000: Ericsson R380.}
Al iniciar el nuevo siglo, Ericsson lanzó el que podría ser considerado el primer Smartphone propiamente dicho. Se trataba del R380, un terminal que costaba cerca de US\$ 700 y el primero en usar el sistema operativo Symbian OS, que contenía un soporte para Bluetooth.
\\ \par
\textbf{2000: SPH-M100.}
De Samsung, fue el primer celular con reproductor de archivos MP3. Su display era un LCD monocromo con capacidad para cinco líneas de texto y gráficos sencillos. Contaba con 64MB de memoria interna y la reproducción se controlaba desde un mando en el cable de los auriculares.
\\ \par
\textbf{Junio 2000: SCH-V200.}
Samsung lanzó al mercado de Corea del Sur el modelo SCH-V200, el primer teléfono con cámara fotográfica incorporada. El dispositivo tomaba fotos de 0,35 megapíxeles y tenía capacidad para tomar 20 fotografías.
\\ \par
\textbf{Noviembre 2000: Sharp J-SH04.}
Sharp Corporation puso a la venta en Japón el J-SH04, un teléfono móvil con una cámara, la cual contaba con un dispositivo de imagen CMOS (Complementary Metal Oxide Semiconductor, por sus siglas en inglés CMOS), (tecnología para evitar menor consumo de energía) con una lente de 0,11 megapíxeles que permitía tomar fotografías y enviarlas a otras personas por correo electrónico.
\\ \par
\textbf{2001: Sony Ericsson T68.}
Fue el primer teléfono móvil en el mundo con pantalla de color.
\\ \par
\textbf{Noviembre 2002: Sanyo SCP-5300.}
Fue el primer dispositivo con cámara fotográfica que llegó a los Estados Unidos. Las novedades en este teléfono con forma de “Concha” eran los ajustes personalizables de la cámara, como uso de flash, control de balance de blancos, auto disparador, zoom digital y filtros como sepia, blanco y negro, negativo, etc.
\\ \par
\textbf{Octubre 2003: Nokia N-Gage.}
Nokia presentó, una apuesta de la empresa con sede en Finlandia por combinar un teléfono móvil con una videoconsola portátil. Si bien fue un fracaso comercial, logró abrir un hueco en la historia de la telefonía móvil dado su carácter único.
\\ \par
\textbf{2003: Palmone Treo 600.}
La empresa Palm lanzó el Treo 600 que contaba con cuatribanda Sistema Global para comunicaciones Móviles(Global System for Mobile communications, por sus siglas en inglés GSM), pantalla de color, navegador de cinco direcciones, teclado retroiluminado y cámara digital.
\\ \par
\textbf{9 de enero del 2007: iPhone.}
Steve Jobs presentó el iPhone de Apple, seis meses después se empieza a vender con mucho éxito en Estados Unidos. Además de llamadas telefónicas, recibir y enviar mensajes y tomar fotos, daba información de la bolsa, tenía correo electrónico, calendario, estado del tiempo, entre otras cosas. Marcó el hito, pues los otros fabricantes comenzaron a copiar las características del iPhone.
\\ \par
\textbf{22 de octubre del 2008: HTC Dream.}
HTC lanzó el Dream, comercializado también como T-Mobile G1 y denominado popularmente Google Phone o Gphone. Según su fabricante, fue el primer dispositivo móvil de comunicación en incorporar el sistema operativo móvil de Google denominado Android.
\\ \par
\textbf{2009: OMNIA HD i8910.}
El primer teléfono celular del mundo capaz de grabar video en alta definición. También poseía soporte DivX, puerto HDMI, conector de 3.5mm, Wi-Fi con DLNA (Digital Living Network Alliance) y GPS.
\\ \par
\textbf{2010: Nokia N8.}
Fue el teléfono celular insignia de Nokia, con una pantalla touchscreen capacitiva de 3.5 pulgadas, cámara de 12 megapíxeles con flash Xenon y captura de video a 720p y 25 cuadros por segundo, salida interfaz Multimedia de Alta Definición (High-Definition Multimedia Interface, por sus siglas en inglés HDMI), radio FM con transmisor, Wi-Fi, GPS y 16GB de memoria interna entre otras características. Fue el primer teléfono inteligente de Nokia en correr el sistema operativo Symbian 3.
\\ \par
\textbf{2010: Sony Ericsson Z1010.}
Fue el primer celular con dos cámaras, dos pantallas y una velocidad que multiplica por dos su rendimiento. Su pantalla de color ofrecía una estupenda visibilidad bajo luz solar directa.
\\ \par
\textbf{2011: iPhone 4S.}
Poseía la misma pantalla de 3.5 pulgadas a 640x960 píxeles de resolución, pero con un procesador dual-core A5 que le proveía el doble de velocidad y le incorpora una cámara de 8 megapíxeles con captura de video 1080p y corre con el renovado iOS5.
\\ \par
\textbf{2011: LG Optimus 3D.}
Fue el primer teléfono celular capaz de leer gráficos 3D. Posee una pantalla LCD capacitiva de 4.3 pulgadas a 480x800 píxeles de resolución, con tecnología 3D sin necesidad de anteojos. Las dos cámaras de 5 megapíxeles en la parte posterior graban video 3D estereoscópico a resolución HD, que se puede publicar en un canal especial de YouTube.
\\ \par
\textbf{2011: Motorola Atrix.}
Es un teléfono inteligente con sistema operativo Android y un procesador NVIDIA Tegra 2 de doble núcleo. Posee una pantalla de 5 megapíxeles con captura de video HD, Wi-Fi, GPS, 1GB de RAM y una batería de 1930mAh.
\\ \par
\textbf{2015: Fijitsu Arrows NX F-04G.}
Es el primer terminal con reconocimiento del iris, un sistema que se utiliza actualmente para desbloquear el teléfono como para aceptar pagos electrónicos. El dispositivo fue presentado por la operadora DoCoMo. \\
