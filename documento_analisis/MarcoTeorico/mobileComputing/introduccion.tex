
La Computación Móvil se refiere al uso de computadoras sin la necesidad de estar conectadas a una red cableada, sino a través de redes inalámbricas o satelitales. Se puede definir la Computación Móvil como los dispositivos que hacen uso de la computación para lograr su funcionamiento, de esta forma se desarrollan las computadoras portátiles, los teléfonos celulares, los cuadernos de notas computarizados, las calculadoras de bolsillo, entre otras. La computación móvil se está volviendo día a día un paradigma tecnológico de uso común, el cual está cambiando la forma en que se realizan las actividades laborales, académicas, de investigación y entretenimiento, como en su momento lo hizo la computación como se conoce hasta hoy \cite{MBDef}. \\

El cómputo móvil implica software, hardware y comunicación móvil. La portabilidad es un aspecto importante del mismo ya que se le conoce como la capacidad de utilizar la capacidad informática sin una ubicación predefinida y/o la conexión a una red para captar datos e información \cite{MBPDO}. \\

El continuo avance de la tecnología y las necesidades básicas de comunicación, han permitido el crecimiento de la computación móvil, como un elemento en la vida cotidiana, en la agilización y optimización de los procesos de los organizadores e instituciones educativos. En los próximos años, los teléfonos inteligentes, serán quienes dirijan el consumo de dispositivos móviles \cite{MBIntro}. \\

