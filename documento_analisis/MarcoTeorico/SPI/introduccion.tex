
Muchas  cosas que resultaban útiles hace más de 30 años hoy en día pueden estar obsoletas, como en el caso de los mapas, esto cambió radicalmente con los avances tecnológicos a inicio de los 2000 con la popularización de la navegación por satélite (Sistemas globales de navegación por satélite, GNSS, por sus siglas en inglés) con el ya famoso GPS, que nos permite determinar la posición de un objeto en toda la tierra \cite{SPIUoc}. \\

Con estos navegadores es más rápido y eficiente llegar a cualquier sitio. Con la llegada del GPS a los dispositivos móviles conocer la ubicación de lugares, ha significado un gran impacto tanto para los usuarios como para las empresas de logística y mensajería \cite{SPIitbs}. \\

Sin embargo, existen díficultades para determinar la ubicación de una persona en espacios interiores ya que la señal de los satélites no es capaz de llegar con la intensidad necesaria a estos y si llegara se necesitarían  tener los mapas de los edificios de forma pública y en un formato adecuado para que un dispositivo móvil fuera capaz de guiarnos por ellos \cite{SPIUoc}. \\

Un sistema de posicionamiento en interiores (Indoor Positioning System, por sus siglas en inglés IPS) es una red de dispositivos utilizados para localizar inalámbricamente objetos o personas dentro de un edificio. En lugar de usar los satélites, un IPS se basa en anclajes de proximidad que localizan etiquetas o proporcionan contexto ambiental a los sensores \cite{SPIDef}. \\

La red de dispositivos se puede crear mediante balizas Bluetooth, de ultrasonidos, de  banda ultra ancha (Ultra Wide Band, por sus siglas en inglés UWB), de infrarrojos y usando otros tipos de sensores dedicados. En cuanto a las aplicaciones, la herramienta puede mejorar la navegación en hospitales, para ayudar a los usuarios a encontrar más fácilmente las áreas a las que desean llegar. También puede emplearse en tiendas de retail, terminales de autobuses, aeropuertos y para temas de logística en una fábrica o en grandes superficies  \cite{SPIitbs}. \\