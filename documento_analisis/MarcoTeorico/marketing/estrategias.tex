
\subsubsection{Conexión a una red WiFi}

En los centros comerciales, es un acto común que un usuario se conecte a una red WiFi de forma gratuita para evitar el consumo de datos, esto da oportunidad al marketing de proximidad para enviar mensajes mientras el usuario permanezca conectado a la red. Otra opción es que para poder usar la red de forma gratuita el usuario complete un registro de esta forma se adquiere más información del usuario \cite{MProx}.\\  

\subsubsection{Cupones digitales en el lugar}
El envío de cupones digitales es una práctica común en el marketing de proximidad, ya sea por  
 Bluetooth o Wi-fi. Puede ser una herramienta poderosa para aprovechar los tiempos de alto tráfico, probar el interés en un nuevo producto o estimular las ventas en un día lento \cite{MProx}.\\ 


\subsubsection{Lectura de código QR}
El código QR (Quick Response barcode) es un código de barras bidimensional que actualmente se ha adaptado como una forma más de conectar con los clientes. Esta estrategia funciona únicamente si los mismos eligen escanear el código con su smartphone, pero requiere menos tiempo y atención que la señalización digital. Los compradores pueden escanear sobre la marcha y conectarse con el contenido en el acto o marcarlo y volver a él más tarde \cite{MProx}.\\ 

\subsubsection{NFC}

NFC son las siglas de Near Field Communication, una tecnología de comunicación inalámbrica que permite hacer pagos con un smartphone, pero también permite interacción entre clientes y marcas. Tiene muy corto alcance, es decir, se hace necesario que los terminales estén prácticamente en contacto \cite{MPNFC}. \\ 

La ventaja sobre el código QR es la comodidad y la velocidad, el NFC elimina los pasos de tener que realizar una foto; sólo es necesario pasar el dispositivo por encima del NFC y ya se cuenta con la información en pantalla por lo que hace falta únicamente imaginarse las campañas relacionadas con las ofertas y descuentos, con entradas a eventos o las promociones de puntos \cite{MPNFC2}. \\

\subsubsection{Beacons}

Un Smartphone que cuenta con una aplicación rastreadora de señales de Beacon abre un mundo de posibilidades para los compradores, cada vez que un usuario entra a una tienda que cuente con estos dispositivos, puede recibir notificaciones o información de todo tipo: cupones de descuentos, catálogos, ofertas e información adicional que pudiera serle de utilidad al momento de hacer compras en el lugar \cite{MProx}.\\ 