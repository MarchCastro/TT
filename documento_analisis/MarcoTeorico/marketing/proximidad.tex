
Se conoce como marketing de proximidad a aquellos sistemas que utilizan tecnologías de localización para comunicarse directamente con los consumidores por medio de sus dispositivos móviles. Entre los usos del marketing de proximidad se encuentran la distribución de multimedia en conciertos, información de aplicaciones sociales y anuncios locales. Este tipo de marketing no se ve limitado únicamente a teléfonos celulares, puede ser implementado tanto en laptops como en tablets que tienen su GPS activo y que de igual manera, pueden ser conectados a tecnologías de proximidad como: NFC (Near Field Communication), Geofencing y Wi-Fi Hotspot \cite{MarketingProx}. \\ \par

Lugares como centros comerciales, tiendas departamentales y grandes puntos de venta donde los consumidores generalmente planean pasar al menos una hora o dos, son áreas privilegiadas para una campaña de marketing de proximidad, debido a que los clientes se encuentran haciendo compras activamente, lo que permite que sean más abiertos a información, promociones y sugerencias de productos \cite{Marketing1}. \\ 