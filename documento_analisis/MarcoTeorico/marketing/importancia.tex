
Según un análisis de The Competitive Intelligence Unit, en México existen 112.8 millones de líneas móviles activas de las cuales el 85\% corresponden a smartphones. Con la cantidad mencionada se puede afirmar que los teléfonos se están convirtiendo en una necesidad básica del siglo XXI, lo cual hace a casi todos los consumidores con un teléfono inteligente potencialmente susceptibles a una campaña de marketing de proximidad, en particular los jóvenes que son más propensos a hacer uso de dichos dispositivos durante sus compras  \cite{Marketing2}.
\\ \par
Entre las principales ventajas que el marketing de proximidad ofrece a los clientes y consumidores se encuentran \cite{Marketing3}: \\
\begin{itemize}
\item Acceso inmediato a la información de los productos/servicios que satisfacerán la necesidad, provista por parte de la empresa.
\item Recepción de promociones, cupones de descuento y reducciones en futuras compras.
\item Mediante el acceso a diferentes motores de búsqueda, se obtiene una mayor facilidad para conocer la calidad de un servicio o producto.
\item Interacción en tiempo real mediante la geolocalización de los clientes, mismo que promueve el aspecto lúdico y atractivo lo cual incita a una participación por parte del consumidor. 
\end{itemize}