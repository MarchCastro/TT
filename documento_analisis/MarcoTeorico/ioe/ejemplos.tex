
Actualmente existen muchos ejemplos para describir la función del IoE, un ejemplo es la aplicación Google Maps, en la cuál podemos conocer zonas de embotellamiento vial, para tomar en cuenta vías alternas y llegar a tiempo a nuestro destino. Esto sólo por mencionar una de las utilidades de esta multifuncional aplicación.
\\ \par
IoT o IoE está en un gran crecimiento gracias a compañías como Intel, Google, Cisco , Microsoft Corp. e IBM (International Business Machines Corp.) poniendo todo su empeño detrás, al igual que la cantidad de dispositivos conectados \cite{IoEOpenMind}. 
\\ \par
En 2012 había 8.700 millones de objetos conectados en todo el mundo, que representaban el 0.6\% de las ``cosas'' en el planeta, según Cisco Systems Inc. en 2013, este número superó los 10 mil millones y Cisco espera que el número de objetos conectados alcance 50 mil millones para 2020 ó el 2.7\% de las ``cosas'' en el mundo, impulsadas por la reducción del precio por conexión y el rápido crecimiento en el número de conexiones de máquina a máquina \cite{IoELiveMint}.
\\ \par
Según ABI Research, Bluetooth, Wi-Fi, ZigBee, Cellular, Identificación por Radiofrecuencia (Radio Frequency Identification, por sus siglas en inglés RFID) y muchas otras tecnologías inalámbricas impulsan el crecimiento del IoE \cite{IoELiveMint}. De acuerdo con la firma de investigación Gartner Inc. IoE creará decenas de millones de nuevos objetos y sensores que recopilen datos en tiempo real, así mismo pronostica que las empresas harán un uso intensivo de la tecnología de Tecnologías de la Información (Information Technology, por sus siglas en inglés IT) y que se venderá una amplia gama de productos en diversos mercados. Entre ellos se incluyen dispositivos médicos avanzados, sensores para la automatización de fábricas y aplicaciones de robótica industrial, sensores minúsculos para aumentar la producción agrícola, sensores para el sector de la automoción y sistemas de monitorización de la integridad de infraestructuras para áreas tan diversas como el transporte por carretera y por ferrocarril, la distribución de agua y el transporte de electricidad, en suma, una lista interminable de productos y servicios \cite{IoEOpenMind}. \\