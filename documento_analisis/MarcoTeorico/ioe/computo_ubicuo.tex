

\subsubsection{ Cómputo ubicuo }

Es entendida como la integración de la informática en el entorno de la persona, de forma que los ordenadores no se perciban como objetos diferenciados, apareciendo en cualquier lugar y en cualquier momento \cite{IOEcu1}.
\\ \par
A diferencia de la computación de escritorio, la computación ubicua puede ocurrir al emplear cualquier dispositivo, en cualquier ubicación y en cualquier formato. El usuario interactúa con la computadora embebida, que puede existir en distintas formas, incluyendo computadoras portátiles, tabletas y terminales en objetos comunes tales como refrigeradores, televisiones o un juego de anteojos. La tecnología subyacente que soporta la computación ubicua incluye el Internet, el middleware, sistemas operativos, código móvil, sensores, microprocesadores, interfaces de usuario, redes, protocolos de comunicación, posicionamiento y ubicación y nuevos materiales \cite{IOEcu1}.
\\ \par
La mejor forma de saber que estamos frente a computación ubicua es que las computadoras están integradas al ambiente de tal forma que nuestra interacción con estas sea casi invisible pero si perceptible, por esto también es llamada Inteligencia Ambiental. Es un ambiente basado en un modelo de interacción en el cual los usuarios están rodeados de un entorno digital consciente de su presencia, sensible al contexto y que puede adaptarse a sus necesidades y hábitos, Además, el cómputo ubicuo es una visión que coloca al ser humano como centro de desarrollo futuro en la sociedad del conocimiento, la información y la tecnología \cite{IOEcu2}.
\\ \par

El papel que juega el Internet en el cómputo ubicuo es primordial: la alta conectividad, la capacidad de interconectar sistemas y acceder a diferentes bases de datos. Los servicios Web nos permiten acceder a todo esto de forma sencilla y estandarizada. Cualquier dispositivo con conectividad a Internet tiene el potencial de realizar peticiones a Web services y de esta forma recibir o actualizar información \cite{IOEcu2}.
\\ \par