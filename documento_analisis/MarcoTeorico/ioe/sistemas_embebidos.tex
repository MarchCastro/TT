
\subsubsection{ Sistemas embebidos }

Para definir un sistema embebido es necesario distinguir cuatro características fundamentales: Hardware (embedded hardware), software (embedded software), inteligencia computacional y ejecución de una o varias tareas en tiempo real (el sistema es predecible y determinista). En este sentido, un sistema embebido se define como un dispositivo electrónico que tiene inteligencia computacional, diseñado para cumplir una o varias tareas relacionadas que se determinan desde el diseño y por lo tanto, son predecibles al ejecutarse en tiempo real y que está integrado por componentes de hardware y software. Los sistemas embebidos representan una evolución de los sistemas electrónicos rígidos, y su principal característica es la posibilidad de ser programados para resolver algorítmicamente un problema determinado. La programación de estos dispositivos puede ser realizada por el diseñador del sistema o por el sistema mismo, lo que da la sensación de razonamiento, es decir, de que el sistema puede aprender por sí mismo \cite{IOESisEmb}.
\\ \par


Cuando se comunican estos sistemas con Internet, se les cambia de nombre por el de sistemas embebidos ``Smart'' o dispositivos "Smart" (Smart TV, Smart Phone). Así, adicional a su función original, envían información relacionada al problema que resuelven (la Smart TV, por ejemplo, podría enviar una bitácora con los canales más vistos y el horario en el que es utilizada). La información emitida por estos sistemas es almacenada en servidores, para procesarla y predecir patrones de conducta de los usuarios o avisar de fallas del sistema al proveedor. Los sistemas embebidos son cada vez más usados en todos los ámbitos de nuestras vidas: Desde un sensor de aparcamiento hasta una cafetera que se comunica con tu celular, a un smartwatch \cite{IOESisEmb}.
\\ \par