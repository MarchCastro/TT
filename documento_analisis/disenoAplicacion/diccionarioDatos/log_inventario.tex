\title{\textbf{
Tabla de diccionario de datos Log\_inventario
}} \\

El cuadro \ref{table:dic-Log} indica la descripción correspondiente de la entidad Log\_inventario.
\label{Entidad-Log_inventario}
\FloatBarrier
\begin{table}[htb]
\setlength\extrarowheight{2pt}
\begin{tabular}{|p{2.5cm}|p{1.5cm}|p{1.5cm}|p{1.5cm}|p{1.5cm}|p{5.5cm}|}
	\hline
	\multicolumn{2}{|c|} 
	{{
		\cellcolor[HTML]{00009B}{\color[HTML]{FFFFFF} Nombre de la relación: }		
	}} &
	\multicolumn{4}{|c|} {{ Log\_inventario }} \\
	\hline
	\multicolumn{2}{|c|} 
	{{
		\cellcolor[HTML]{00009B}{\color[HTML]{FFFFFF} Objetivo: }		
	}} &
	\multicolumn{4}{|c|} {{ Tabla que almacenará los registros del inventario de un producto. }} \\
	\hline
	\rowcolor[HTML]{3166FF} 
	{\color[HTML]{FFFFFF} Campo }  & 
	{\color[HTML]{FFFFFF} Tipo de dato } & 
	{\color[HTML]{FFFFFF} Tamaño } & 
	{\color[HTML]{FFFFFF} Constraint } & 
	{\color[HTML]{FFFFFF} No nulo } & 
	{\color[HTML]{FFFFFF} Descripción } \\ 
	\hline
	\cellcolor[HTML]{9B9B9B}{\color[HTML]{FFFFFF} id\_log\_inventario } &
	bigint &
	20 &
	PK &
	X  & 
	Identificador del registro en inventario.   \\ 
	\hline
	
	\cellcolor[HTML]{9B9B9B}{\color[HTML]{FFFFFF} fk\_id\_producto } &
	bigint &
	20 &
	FK &
	X &
	Identificador del producto.  \\ 
	\hline
	
	\cellcolor[HTML]{9B9B9B}{\color[HTML]{FFFFFF} inventario\_inicial } &
	smallint &
	5 &
	&
	X & 
	Valor inicial del total de stock en inventario del producto.   \\ 
	\hline

	\cellcolor[HTML]{9B9B9B}{\color[HTML]{FFFFFF} fecha\_ingreso } &
	timestamp &
	 &
	&
	X  & 
	Fecha de ingreso de nuevo stock del producto.   \\ 
	\hline
	
	\cellcolor[HTML]{9B9B9B}{\color[HTML]{FFFFFF} fecha\_pedido } &
	timestamp &
	 &
	&
	X  & 
	Fecha en la que se realizó el pedido.   \\ 
	\hline
				
\end{tabular}
\caption{Tabla de diccionario de datos Log\_inventario. }
\label{table:dic-Log}
\end{table}
\FloatBarrier