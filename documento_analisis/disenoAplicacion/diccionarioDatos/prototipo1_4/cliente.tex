\title{\textbf{
Tabla de diccionario de datos Cliente
}} \\

El cuadro \ref{table:dic4-Cliente} indica la descripción correspondiente a las modificaciones de la entidad Cliente.
\label{Entidad-Cliente}
\FloatBarrier
\begin{table}[htb]
\setlength\extrarowheight{2pt}
\begin{tabular}{|p{2.5cm}|p{1.5cm}|p{1.5cm}|p{1.5cm}|p{1.5cm}|p{5.5cm}|}
	\hline
	\multicolumn{2}{|c|} 
	{{
		\cellcolor[HTML]{00009B}{\color[HTML]{FFFFFF} Nombre de la relación: }		
	}} &
	\multicolumn{4}{|c|} {{ Cliente }} \\
	\hline
	\multicolumn{2}{|c|} 
	{{
		\cellcolor[HTML]{00009B}{\color[HTML]{FFFFFF} Objetivo: }		
	}} &
	\multicolumn{4}{|c|} {{ Tabla que almacenará los datos especificos de un cliente. }} \\
	\hline
	\rowcolor[HTML]{3166FF} 
	{\color[HTML]{FFFFFF} Campo }  & 
	{\color[HTML]{FFFFFF} Tipo de dato } & 
	{\color[HTML]{FFFFFF} Tamaño } & 
	{\color[HTML]{FFFFFF} Constraint } & 
	{\color[HTML]{FFFFFF} No nulo } & 
	{\color[HTML]{FFFFFF} Descripción } \\ 
	\hline
	\cellcolor[HTML]{9B9B9B}{\color[HTML]{FFFFFF} id\_cliente } &
	bigint &
	20 &
	PK &
	X  & 
	Identificador del cliente.   \\ 
	\hline 
	\cellcolor[HTML]{9B9B9B}{\color[HTML]{FFFFFF} fk\_id\_persona } &
	bigint &
	20 &
	FK &
	X  & 
	Identificador de la persona.   \\ 
	\hline
	\cellcolor[HTML]{9B9B9B}{\color[HTML]{FFFFFF} fk\_id\_nivel } &
	smallint &
	5 &
	FK &
	X  & 
	Identificador del nivel del cliente.   \\ 
	\hline
	\cellcolor[HTML]{9B9B9B}{\color[HTML]{FFFFFF} no\_hijo } &
	tinyInt &
	2 &
	 &
	 & 
	Número de hijos que tiene el cliente.   \\ 
	\hline
	\cellcolor[HTML]{9B9B9B}{\color[HTML]{FFFFFF} gustos } &
	varchar[] &
	100 &
	 &
	X  & 
	Gustos de la persona almacenados en un arreglo de longitud 100 cada elemento.   \\ 
	\hline
	\cellcolor[HTML]{9B9B9B}{\color[HTML]{FFFFFF} ultima\_actualizacion\_nivel } &
	timestamp &
	 &
	 &
	X  & 
	Fecha de la última actualización del nivel del cliente. \\ 
	\hline
	\cellcolor[HTML]{9B9B9B}{\color[HTML]{FFFFFF} fecha\_registro } &
	timestamp &
	 &
	 &
	X  & 
	Fecha en la que se registro el cliente en Sapphire. \\ 
	\hline
	\cellcolor[HTML]{9B9B9B}{\color[HTML]{FFFFFF} puntaje } &
	smallint &
	 2 &
	 &
	X  & 
	Puntaje que el cliente tiene actualmente, este puntaje se obtiene por las compras o acciones que realice. \\ 
	\hline
	\cellcolor[HTML]{9B9B9B}{\color[HTML]{FFFFFF} grupo } &
	smallint &
	 2 &
	 &
	 & 
	Grupo al que pertenece el cliente. \\ 
	\hline
	\cellcolor[HTML]{9B9B9B}{\color[HTML]{FFFFFF} posicion } &
	point &
	 &
	 &
	 & 
	Coordenada en que se encuentra el cliente dentro de la gráfica de clusters. \\ 
	\hline
	\cellcolor[HTML]{9B9B9B}{\color[HTML]{FFFFFF} permisos } &
	json &
	  &
	 &
	X  & 
	Lista de permisos que puede otorgar el cliente al sistema para acceder a su información: edad, estado civil, número de hijos, historial de compras, productos favoritos y recibir recompensas presenciales. Valor default : '\{``edad": false,``estado\_civil": false,``hijos": false, ``compras":false,``favoritos": false, ``recom\_presenciales": false\}'.   \\ 
	\hline
	\cellcolor[HTML]{9B9B9B}{\color[HTML]{FFFFFF} thetas } &
	double[] &
	  &
	 &
	 & 
	Vector de pesos de un usuario el cuál permite hacer la predicción de una calificación sobre un producto.   \\ 
	\hline
	
	
\end{tabular}

\caption{Tabla de diccionario de datos Cliente. }
\label{table:dic4-Cliente}
\end{table}
\FloatBarrier

