\title{\textbf{
Tabla de diccionario de datos Producto
}}\\

El cuadro \ref{table:dic4-Producto} indica la descripción correspondiente de la entidad Producto.
\label{Entidad-Producto}
\FloatBarrier
\begin{table}[htb]
\setlength\extrarowheight{2pt}
\begin{tabular}{|p{2.5cm}|p{1.5cm}|p{1.5cm}|p{1.5cm}|p{1.5cm}|p{5.5cm}|}
	\hline
	\multicolumn{2}{|c|} 
	{{
		\cellcolor[HTML]{00009B}{\color[HTML]{FFFFFF} Nombre de la relación: }		
	}} &
	\multicolumn{4}{|c|} {{ Producto }} \\
	\hline
	\multicolumn{2}{|c|} 
	{{
		\cellcolor[HTML]{00009B}{\color[HTML]{FFFFFF} Objetivo: }		
	}} &
	\multicolumn{4}{|c|} {{ Tabla que almacenará la información general de un producto. }} \\
	\hline
	\rowcolor[HTML]{3166FF} 
	{\color[HTML]{FFFFFF} Campo }  & 
	{\color[HTML]{FFFFFF} Tipo de dato } & 
	{\color[HTML]{FFFFFF} Tamaño } & 
	{\color[HTML]{FFFFFF} Constraint } & 
	{\color[HTML]{FFFFFF} No nulo } & 
	{\color[HTML]{FFFFFF} Descripción } \\ 
	\hline
	\cellcolor[HTML]{9B9B9B}{\color[HTML]{FFFFFF} id\_producto } &
	bigint &
	20 &
	PK &
	X  & 
	Identificador del producto.   \\ 
	\hline
	\cellcolor[HTML]{9B9B9B}{\color[HTML]{FFFFFF} nombre } &
	varchar &
	150 &
	 &
	X  & 
	Nombre del producto.   \\ 
	\hline		
	
	\cellcolor[HTML]{9B9B9B}{\color[HTML]{FFFFFF} precio\_venta } &
	decimal &
	10, 2 &
	 &
	X  & 
	Precio de venta del producto.   \\ 
	\hline		
	
	\cellcolor[HTML]{9B9B9B}{\color[HTML]{FFFFFF} costo } &
	decimal &
	10, 2 &
	 &
	X  & 
	Costo del producto.   \\ 
	\hline		
	
	\cellcolor[HTML]{9B9B9B}{\color[HTML]{FFFFFF} descripcion } &
	text &
	 &
	 &
	X  & 
	Breve descripción del producto.   \\ 
	\hline		
	
	\cellcolor[HTML]{9B9B9B}{\color[HTML]{FFFFFF} fk\_id\_departamento } &
	bigint &
	20 &
	FK &
	X  & 
	Identificador del departamento al que pertenece el producto.   \\ 
	\hline
	
	\cellcolor[HTML]{9B9B9B}{\color[HTML]{FFFFFF} fk\_id\_categoria } &
	bigint &
	20 &
	FK &
	X  & 
	Identificador de la categoría a la que pertenece el producto.   \\ 
	\hline
	
	\cellcolor[HTML]{9B9B9B}{\color[HTML]{FFFFFF} fk\_id\_marca } &
	integer &
	10 &
	FK &
	X  & 
	Identificador de la marca de producto.   \\ 
	\hline
	
	\cellcolor[HTML]{9B9B9B}{\color[HTML]{FFFFFF} fk\_id\_promocion } &
	smallint &
	5 &
	FK &
	  & 
	Identificador de una posible promoción que pueda tener el producto.   \\ 
	\hline
	
	\cellcolor[HTML]{9B9B9B}{\color[HTML]{FFFFFF} stock } &
	integer &
	10 &
	 &
	X  & 
	Número actual de productos disponibles para venta.   \\ 
	\hline	
	
	\cellcolor[HTML]{9B9B9B}{\color[HTML]{FFFFFF} no\_ventas } &
	integer &
	10 &
	 &
	X  & 
	Número de veces que se ha vendido este producto.   \\ 
	\hline	
	
	\cellcolor[HTML]{9B9B9B}{\color[HTML]{FFFFFF} x } &
	double[] &
	 &
	 &
	 & 
	Vector de pesos de un producto el cuál representa las características del mismo.   \\ 
	\hline
	
	\cellcolor[HTML]{9B9B9B}{\color[HTML]{FFFFFF} mu } &
	double &
	 &
	 &
	 & 
	Calificación media de un producto.   \\ 
	\hline
\end{tabular}
\caption{Tabla de diccionario de datos Producto.}
\label{table:dic4-Producto}
\end{table}
\FloatBarrier

