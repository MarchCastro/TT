\title{\textbf{
Tabla de diccionario de datos Persona
}}\\

El cuadro \ref{table:dic4-Persona} indica la descripción correspondiente de la entidad Persona.
\label{Entidad-Persona}
\FloatBarrier
\begin{table}[htb]
\setlength\extrarowheight{2pt}
\begin{tabular}{|p{2.5cm}|p{1.5cm}|p{1.5cm}|p{1.5cm}|p{1.5cm}|p{5.5cm}|}
	\hline
	\multicolumn{2}{|c|} 
	{{
		\cellcolor[HTML]{00009B}{\color[HTML]{FFFFFF} Nombre de la relación: }		
	}} &
	\multicolumn{4}{|c|} {{ Persona }} \\
	\hline
	\multicolumn{2}{|c|} 
	{{
		\cellcolor[HTML]{00009B}{\color[HTML]{FFFFFF} Objetivo: }		
	}} &
	\multicolumn{4}{|c|} {{ Tabla que almacenará los datos generales de una persona. }} \\
	\hline
	\rowcolor[HTML]{3166FF} 
	{\color[HTML]{FFFFFF} Campo }  & 
	{\color[HTML]{FFFFFF} Tipo de dato } & 
	{\color[HTML]{FFFFFF} Tamaño } & 
	{\color[HTML]{FFFFFF} Constraint } & 
	{\color[HTML]{FFFFFF} No nulo } & 
	{\color[HTML]{FFFFFF} Descripción } \\ 
	\hline
	\cellcolor[HTML]{9B9B9B}{\color[HTML]{FFFFFF} id\_persona } &
	bigint &
	20 &
	PK &
	X  & 
	Identificador de la persona.   \\ 
	\hline
	\cellcolor[HTML]{9B9B9B}{\color[HTML]{FFFFFF} nombre } &
	varchar &
	100 &
	 &
	X  & 
	Nombre(s) de la persona.   \\ 
	\hline
	\cellcolor[HTML]{9B9B9B}{\color[HTML]{FFFFFF} apellido\_paterno } &
	varchar &
	50 &
	 &
	X  & 
	Apellido paterno de la persona.   \\ 
	\hline
	\cellcolor[HTML]{9B9B9B}{\color[HTML]{FFFFFF} apellido\_materno } &
	varchar &
	50 &
	 &
	X  & 
	Apellido materno de la persona. \\ 
	\hline
	\cellcolor[HTML]{9B9B9B}{\color[HTML]{FFFFFF} email } &
	varchar &
	80 &
	 &
	X  & 
	Email de la persona, se conforma de la siguiente expresión regular: 
	\verb/^[_a-z0-9-]+(.[_a-z0$/  
	\verb/-9-]+)*@[a-z0-9-]+$/   
	\verb/[a-z0-9-] +)*(.[a-z]{2,4})$/ . \\ 
	\hline
	\cellcolor[HTML]{9B9B9B}{\color[HTML]{FFFFFF} contrasena } &
	varchar &
	50 &
	 &
	X  & 
	Contraseña de la persona.   \\ 
	\hline
	\cellcolor[HTML]{9B9B9B}{\color[HTML]{FFFFFF} estado\_civil } &
	char &
	1 &
	 &
	  & 
	Estado civil actual de la persona.   \\ 
	\hline
	\cellcolor[HTML]{9B9B9B}{\color[HTML]{FFFFFF} fecha\_nacimiento } &
	datetime &
	 &
	 &
	 & 
	Fecha de nacimiento de la persona.   \\ 
	\hline
	\cellcolor[HTML]{9B9B9B}{\color[HTML]{FFFFFF} fk\_id\_direccion } &
	integer &
	10 &
	FK &
	X & 
	Identificador de la dirección de la persona.   \\ 
	\hline
	\cellcolor[HTML]{9B9B9B}{\color[HTML]{FFFFFF} telefono } &
	varchar &
	20 &
	 &
	 & 
	Teléfono de contacto principal de la persona.   \\ 
	\hline
	\cellcolor[HTML]{9B9B9B}{\color[HTML]{FFFFFF} sexo } &
	tinyint &
	1 &
	 &
	X & 
	Sexo masculino(0) o femenino(1) de la persona.   \\ 
	\hline
	\cellcolor[HTML]{9B9B9B}{\color[HTML]{FFFFFF} imagen } &
	varchar &
	250 &
	 &
	 & 
	Url que apunta a la dirección donde se encuentra almacenada una foto de la persona.   \\ 
	\hline
	\cellcolor[HTML]{9B9B9B}{\color[HTML]{FFFFFF} token\_fcm } &
	varchar &
	200 &
	 &
	 & 
	Token de registro generado por Firebase Cloud Messaging para la instancia de las aplicaciones cliente.   \\ 
	\hline
	
\end{tabular}
\caption{Tabla de diccionario de datos Persona. }
\label{table:dic4-Persona}
\end{table}
\FloatBarrier

