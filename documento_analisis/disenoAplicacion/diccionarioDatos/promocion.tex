\title{\textbf{
Tabla de diccionario de datos Promoción
}}\\

El cuadro \ref{table:dic-Promocion} indica la descripción correspondiente de la entidad Promoción.
\label{Entidad-Promocion}
\FloatBarrier
\begin{table}[htb]
\setlength\extrarowheight{2pt}
\begin{tabular}{|p{2.5cm}|p{1.5cm}|p{1.5cm}|p{1.5cm}|p{1.5cm}|p{5.5cm}|}
	\hline
	\multicolumn{2}{|c|} 
	{{
		\cellcolor[HTML]{00009B}{\color[HTML]{FFFFFF} Nombre de la relación: }		
	}} &
	\multicolumn{4}{|c|} {{ Promocion }} \\
	\hline
	\multicolumn{2}{|c|} 
	{{
		\cellcolor[HTML]{00009B}{\color[HTML]{FFFFFF} Objetivo: }		
	}} &
	\multicolumn{4}{|p{10cm}|} {{ Tabla que almacenará la información sobre las promociones que se relacionan con los productos y/o compras. }} \\
	\hline
	\rowcolor[HTML]{3166FF} 
	{\color[HTML]{FFFFFF} Campo }  & 
	{\color[HTML]{FFFFFF} Tipo de dato } & 
	{\color[HTML]{FFFFFF} Tamaño } & 
	{\color[HTML]{FFFFFF} Constraint } & 
	{\color[HTML]{FFFFFF} No nulo } & 
	{\color[HTML]{FFFFFF} Descripción } \\ 
	\hline
	\cellcolor[HTML]{9B9B9B}{\color[HTML]{FFFFFF} id\_promocion } &
	smallint &
	5 &
	PK &
	X  & 
	Identificador de la promoción.   \\ 
	\hline
	
	\cellcolor[HTML]{9B9B9B}{\color[HTML]{FFFFFF} gratificacion } &
	smallint &
	5 &
	 &
	 &
	Bonificación que tiene el producto o que se puede obtener al compralo.  \\ 
	\hline
	
	\cellcolor[HTML]{9B9B9B}{\color[HTML]{FFFFFF} producto\_gratis } &
	smallint &
	5 &
	&
	 & 
	Indicada si el producto o la compra del mismo se puede obtener de forma gratuita.   \\ 
	\hline
	
	\cellcolor[HTML]{9B9B9B}{\color[HTML]{FFFFFF} porcentaje } &
	smallint &
	5 &
	 &
	  & 
	Porcentaje de descuento que puede tener el producto.   \\ 
	\hline

	\cellcolor[HTML]{9B9B9B}{\color[HTML]{FFFFFF} fecha\_inicio } &
	timestamp &
	 &
	&
	X  & 
	Fecha a partir de la cual la promoción es válida.   \\ 
	\hline
	
	\cellcolor[HTML]{9B9B9B}{\color[HTML]{FFFFFF} fecha\_fin } &
	timestamp &
	 &
	&
	X  & 
	Fecha a partir de la cual la promoción deja de ser válida.   \\ 
	\hline
				
\end{tabular}
\caption{Tabla de diccionario de datos Promoción. }
\label{table:dic-Promocion}
\end{table}
\FloatBarrier

