\title{\textbf{
Tabla de diccionario de datos Imagen\_producto 
}} \\

El cuadro \ref{table:dic-ImagenPro} indica la descripción correspondiente de la entidad Imagen\_producto.
\label{Entidad-Imagen_producto}
\FloatBarrier
\begin{table}[htb]
\setlength\extrarowheight{2pt}
\begin{tabular}{|p{2.5cm}|p{1.5cm}|p{1.5cm}|p{1.5cm}|p{1.5cm}|p{5.5cm}|}
	\hline
	\multicolumn{2}{|c|}
	{{
		\cellcolor[HTML]{00009B}{\color[HTML]{FFFFFF} Nombre de la relación: }		
	}} &
	\multicolumn{4}{|c|} {{  Imagen\_producto }} \\
	\hline
	\multicolumn{2}{|c|} 
	{{
		\cellcolor[HTML]{00009B}{\color[HTML]{FFFFFF} Objetivo: }		
	}} &
	\multicolumn{4}{|c|} {{ Tabla que almacenará las rutas de las imágenes de productos. }} \\
	\hline
	\rowcolor[HTML]{3166FF}
	{\color[HTML]{FFFFFF} Campo }  & 
	{\color[HTML]{FFFFFF} Tipo de dato } & 
	{\color[HTML]{FFFFFF} Tamaño } & 
	{\color[HTML]{FFFFFF} Constraint } & 
	{\color[HTML]{FFFFFF} No nulo } & 
	{\color[HTML]{FFFFFF} Descripción } \\ 
	\hline
	\cellcolor[HTML]{9B9B9B}{\color[HTML]{FFFFFF} id\_imagen\_producto } &
	bigint &
	20 &
	PK &
	X  & 
	Identificador de la imagen del producto. \\
	\hline
	\cellcolor[HTML]{9B9B9B}{\color[HTML]{FFFFFF} ruta } &
	varchar &
	250 &
	 &
	X  & 
	Ruta en el servidor de la ubicación de la imagen. \\
	\hline
\end{tabular}
\caption{Tabla de diccionario de datos Imagen\_producto. }
\label{table:dic-ImagenPro}
\end{table}
\FloatBarrier

