\title{\textbf{
Tabla de diccionario de datos Compra
}} \\

El cuadro \ref{table:dic-Compra} indica la descripción correspondiente de la entidad Compra.
\label{Entidad-Compra}
\FloatBarrier
\begin{table}[htb]
\setlength\extrarowheight{2pt}
\begin{tabular}{|p{2.5cm}|p{1.5cm}|p{1.5cm}|p{1.5cm}|p{1.5cm}|p{5.5cm}|}
	\hline
	\multicolumn{2}{|c|} 
	{{
		\cellcolor[HTML]{00009B}{\color[HTML]{FFFFFF} Nombre de la relación: }		
	}} &
	\multicolumn{4}{|c|} {{ Compra }} \\
	\hline
	\multicolumn{2}{|c|} 
	{{
		\cellcolor[HTML]{00009B}{\color[HTML]{FFFFFF} Objetivo: }		
	}} &
	\multicolumn{4}{|c|} {{ Tabla que almacenará los registros del inventario de un producto. }} \\
	\hline
	\rowcolor[HTML]{3166FF} 
	{\color[HTML]{FFFFFF} Campo }  & 
	{\color[HTML]{FFFFFF} Tipo de dato } & 
	{\color[HTML]{FFFFFF} Tamaño } & 
	{\color[HTML]{FFFFFF} Constraint } & 
	{\color[HTML]{FFFFFF} No nulo } & 
	{\color[HTML]{FFFFFF} Descripción } \\ 
	\hline
	\cellcolor[HTML]{9B9B9B}{\color[HTML]{FFFFFF} id\_compra } &
	bigint &
	20 &
	PK &
	X  & 
	Identificador de la compra.   \\ 
	\hline
	
	\cellcolor[HTML]{9B9B9B}{\color[HTML]{FFFFFF} fk\_id\_producto } &
	bigint &
	20 &
	FK &
	X &
	Identificador del producto.  \\ 
	\hline
	
	\cellcolor[HTML]{9B9B9B}{\color[HTML]{FFFFFF} fk\_id\_cliente } &
	bigint &
	20 &
	FK &
	X &
	Identificador del cliente.  \\ 
	\hline
	
	\cellcolor[HTML]{9B9B9B}{\color[HTML]{FFFFFF} fk\_id\_promocion } &
	smallint &
	5 &
	FK &
	 &
	Identificador de una posible promoción.  \\ 
	\hline
	
	\cellcolor[HTML]{9B9B9B}{\color[HTML]{FFFFFF} fecha\_compra } &
	timestamp &
	 &
	&
	X  & 
	Fecha en que se realizó la compra.   \\ 
	\hline
	
	\cellcolor[HTML]{9B9B9B}{\color[HTML]{FFFFFF} recomendando } &
	char &
	1 &
	&
	X & 
	Indica si el producto comprado ha sido recomendado por el sistema o no.   \\ 
	\hline
	
	\cellcolor[HTML]{9B9B9B}{\color[HTML]{FFFFFF} calificacion\_producto } &
	smallint &
	5 &
	 &
	&
	Calificación que el cliente otorga al producto comprado.  \\ 
	\hline
	
	\cellcolor[HTML]{9B9B9B}{\color[HTML]{FFFFFF} no\_ticket } &
	varchar &
	150 &
	 &
	&
	Número de ticket generado a partir de las iniciales de la tienda, la fecha de compra en formato Ymd, identificador del cliente y un número de 8 digitos aleatorio.  \\ 
	\hline

	%\cellcolor[HTML]{9B9B9B}{\color[HTML]{FFFFFF} fecha\_pedido } &
	%timestamp &
	% &
	%&
	%X  & 
	%Fecha en la que se realizó el pedido.   \\ 
	%\hline
				
\end{tabular}

\caption{Tabla de diccionario de datos Compra. }
\label{table:dic-Compra}
\end{table}
\FloatBarrier



