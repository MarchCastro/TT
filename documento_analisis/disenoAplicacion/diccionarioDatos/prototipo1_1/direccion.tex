\title{\textbf{
Tabla de diccionario de datos Dirección
}}\\

El cuadro \ref{table:dic-Direccion2} indica la descripción correspondiente de la entidad Drección del prototipo 1.2.
\label{Entidad-Direccion}
\FloatBarrier
\begin{table}[htb]
\setlength\extrarowheight{2pt}
\begin{tabular}{|p{3.2cm}|p{1.5cm}|p{1.5cm}|p{1.5cm}|p{1.5cm}|p{5.5cm}|}
	\hline
	\multicolumn{2}{|c|} 
	{{
		\cellcolor[HTML]{00009B}{\color[HTML]{FFFFFF} Nombre de la relación: }		
	}} &
	\multicolumn{4}{|c|} {{ Dirección }} \\
	\hline
	\multicolumn{2}{|c|} 
	{{
		\cellcolor[HTML]{00009B}{\color[HTML]{FFFFFF} Objetivo: }		
	}} &
	\multicolumn{4}{|c|} {{ Tabla que almacenará la dirección de una persona o tienda. }} \\
	\hline
	\rowcolor[HTML]{3166FF} 
	{\color[HTML]{FFFFFF} Campo }  & 
	{\color[HTML]{FFFFFF} Tipo de dato } & 
	{\color[HTML]{FFFFFF} Tamaño } & 
	{\color[HTML]{FFFFFF} Constraint } & 
	{\color[HTML]{FFFFFF} No nulo } & 
	{\color[HTML]{FFFFFF} Descripción } \\ 
	\hline
	\cellcolor[HTML]{9B9B9B}{\color[HTML]{FFFFFF} id\_direccion } &
	bigint &
	20 &
	PK &
	X  & 
	Identificador de la dirección del cliente.   \\ 
	\hline
	\cellcolor[HTML]{9B9B9B}{\color[HTML]{FFFFFF} colonia } &
	varchar &
	100 &
	 &
	X  & 
	Nombre de las colonias dentro pertenecientes a un código postal.   \\ 
	\hline
	\cellcolor[HTML]{9B9B9B}{\color[HTML]{FFFFFF} delegacion\_municipio } &
	varchar &
	50 &
	 &
	  & 
	Nombre de la delegación perteneciente a la Ciudad de México o municipio perteneciente a un estado.\\ 
	\hline
	
	\cellcolor[HTML]{9B9B9B}{\color[HTML]{FFFFFF} ciudad\_estado } &
	varchar &
	50 &
	 &
	X  & 
	Nombre de la ciudad(Ciudad de México) o estado perteneciente a la República Mexicana. \\ 
	\hline
	
	\cellcolor[HTML]{9B9B9B}{\color[HTML]{FFFFFF} codigo\_postal } &
	varchar &
	5 &
	 &
	X  & 
	Código postal dentro de la República Mexicana. \\ 
	\hline
	
	\cellcolor[HTML]{9B9B9B}{\color[HTML]{FFFFFF} calle } &
	varchar &
	100 &
	 &
	X  & 
	Nombre de la calle perteneciente a la dirección. \\ 
	\hline
	
\end{tabular}
\caption{Tabla de diccionario de datos Dirección del prototipo 1.2. }
\label{table:dic-Direccion2}
\end{table}
\FloatBarrier

