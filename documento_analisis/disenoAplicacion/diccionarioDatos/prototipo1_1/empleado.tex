\title{\textbf{
Tabla de diccionario de datos Empleado
}}\\

El cuadro \ref{table:dic-Empleado2} indica la descripción correspondiente de la entidad Empleado del prototipo 1.1.
\label{Entidad-Empleado}
\FloatBarrier
\begin{table}[htb]
\setlength\extrarowheight{2pt}
\begin{tabular}{|p{2.5cm}|p{1.5cm}|p{1.5cm}|p{1.5cm}|p{1.5cm}|p{5.5cm}|}
	\hline
	\multicolumn{2}{|c|} 
	{{
		\cellcolor[HTML]{00009B}{\color[HTML]{FFFFFF} Nombre de la relación: }		
	}} &
	\multicolumn{4}{|c|} {{ Empleado }} \\
	\hline
	\multicolumn{2}{|c|} 
	{{
		\cellcolor[HTML]{00009B}{\color[HTML]{FFFFFF} Objetivo: }		
	}} &
	\multicolumn{4}{|c|} {{ Tabla que almacenará los datos especificos de un empleado. }} \\
	\hline
	\rowcolor[HTML]{3166FF} 
	{\color[HTML]{FFFFFF} Campo }  & 
	{\color[HTML]{FFFFFF} Tipo de dato } & 
	{\color[HTML]{FFFFFF} Tamaño } & 
	{\color[HTML]{FFFFFF} Constraint } & 
	{\color[HTML]{FFFFFF} No nulo } & 
	{\color[HTML]{FFFFFF} Descripción } \\ 
	\hline
	\cellcolor[HTML]{9B9B9B}{\color[HTML]{FFFFFF} id\_empleado } &
	integer &
	10 &
	PK &
	X  & 
	Identificador del empleado.   \\ 
	\hline
	\cellcolor[HTML]{9B9B9B}{\color[HTML]{FFFFFF} fk\_id\_persona } &
	bigint &
	20 &
	FK &
	X  & 
	Identificador de la persona.   \\ 
	\hline
	\cellcolor[HTML]{9B9B9B}{\color[HTML]{FFFFFF} fk\_id\_departamento } &
	bigint &
	20 &
	FK &
	X  & 
	Identificador del departamento al que pertecene el empleado.   \\ 
	\hline
	\cellcolor[HTML]{9B9B9B}{\color[HTML]{FFFFFF} rol } &
	char &
	1 &
	 &
	X  & 
	Tipo de rol que tiene el empleado (vendedor/administrador).   \\ 
	\hline
	\cellcolor[HTML]{9B9B9B}{\color[HTML]{FFFFFF} nombre\_usuario } &
	varchar &
	50 &
	 &
	X  & 
	Nombre de usuario con el que se identifica el empleado.   \\ 
	
		
\end{tabular}

\caption{Tabla de diccionario de datos Empleado del prototipo 1.1 . }
\label{table:dic-Empleado2}

\end{table}
\FloatBarrier

