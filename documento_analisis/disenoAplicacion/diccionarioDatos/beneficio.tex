\title{\textbf{
Tabla de diccionario de datos Beneficio
}} \\

El cuadro \ref{table:dic-Beneficio} indica la descripción correspondiente de la entidad Beneficio.
\label{Entidad-Beneficio}
\FloatBarrier
\begin{table}[htb]
\setlength\extrarowheight{2pt}
\begin{tabular}{|p{2.5cm}|p{1.5cm}|p{1.5cm}|p{1.5cm}|p{1.5cm}|p{5.5cm}|}
	\hline
	\multicolumn{2}{|c|} 
	{{
		\cellcolor[HTML]{00009B}{\color[HTML]{FFFFFF} Nombre de la relación: }		
	}} &
	\multicolumn{4}{|c|} {{ Beneficio }} \\
	\hline
	\multicolumn{2}{|c|} 
	{{
		\cellcolor[HTML]{00009B}{\color[HTML]{FFFFFF} Objetivo: }		
	}} &
	\multicolumn{4}{|c|} {{ Tabla que almacenará las marcas que existen en la tienda. }} \\
	\hline
	\rowcolor[HTML]{3166FF} 
	{\color[HTML]{FFFFFF} Campo }  & 
	{\color[HTML]{FFFFFF} Tipo de dato } & 
	{\color[HTML]{FFFFFF} Tamaño } & 
	{\color[HTML]{FFFFFF} Constraint } & 
	{\color[HTML]{FFFFFF} No nulo } & 
	{\color[HTML]{FFFFFF} Descripción } \\ 
	\hline
	\cellcolor[HTML]{9B9B9B}{\color[HTML]{FFFFFF} id\_beneficio } &
	smallint &
	5 &
	PK &
	X  & 
	Identificador del beneficio. \\ 
	\hline
	\cellcolor[HTML]{9B9B9B}{\color[HTML]{FFFFFF} descripcion } &
	text &
	 &
	 &
	X  & 
	Descripción del beneficio.   \\ 
	\hline		

	\cellcolor[HTML]{9B9B9B}{\color[HTML]{FFFFFF} porcentaje } &
	smallint &
	5 &
	 &
	X  & 
	Porcentaje de descuento.   \\ 
	\hline		
	\cellcolor[HTML]{9B9B9B}{\color[HTML]{FFFFFF} gratificacion  } &
	smallint &
	5 &
	 &
	X  & 
	Porcentaje del precio del producto que se otorgará a un cliente como recompensa de su compra.  \\ 
	\hline	
	\cellcolor[HTML]{9B9B9B}{\color[HTML]{FFFFFF} producto\_gratis  } &
	char &
	1 &
	 &
	X  & 
	Indica si el beneficio incluye un producto gratis.   \\ 
	\hline	
			
\end{tabular}

\caption{Tabla de diccionario de datos Beneficio. }
\label{table:dic-Beneficio}
\end{table}
\FloatBarrier



