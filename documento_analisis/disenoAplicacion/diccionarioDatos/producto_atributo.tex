\title{\textbf{
Tabla de diccionario de datos Producto\_atributo
}}\\

El cuadro \ref{table:dic-ProductoAtributo} indica la descripción correspondiente de la entidad Producto\_atributo.
\label{Entidad-Pruducto_atributo}
\FloatBarrier
\begin{table}[htb]
\setlength\extrarowheight{2pt}
\begin{tabular}{|p{2.5cm}|p{1.5cm}|p{1.5cm}|p{1.5cm}|p{1.5cm}|p{5.5cm}|}
	\hline
	\multicolumn{2}{|c|}
	{{
		\cellcolor[HTML]{00009B}{\color[HTML]{FFFFFF} Nombre de la relación: }		
	}} &
	\multicolumn{4}{|c|} {{  Producto\_atributo }} \\
	\hline
	\multicolumn{2}{|c|} 
	{{
		\cellcolor[HTML]{00009B}{\color[HTML]{FFFFFF} Objetivo: }		
	}} &
	\multicolumn{4}{|c|} {{ Tabla que almacenará la relación entre un producto y sus atributos. }} \\
	\hline
	\rowcolor[HTML]{3166FF}
	{\color[HTML]{FFFFFF} Campo }  & 
	{\color[HTML]{FFFFFF} Tipo de dato } & 
	{\color[HTML]{FFFFFF} Tamaño } & 
	{\color[HTML]{FFFFFF} Constraint } & 
	{\color[HTML]{FFFFFF} No nulo } & 
	{\color[HTML]{FFFFFF} Descripción } \\ 
	\hline
	\cellcolor[HTML]{9B9B9B}{\color[HTML]{FFFFFF} fk\_id\_producto } &
	bigint &
	20 &
	PK, FK &
	X  & 
	Identificador del producto. \\
	\hline
	\cellcolor[HTML]{9B9B9B}{\color[HTML]{FFFFFF} fk\_id\_atributo } &
	integer &
	10 &
	PK, FK &
	X  & 
	Identificador del atributo. \\
	\hline
\end{tabular}
\caption{Tabla de diccionario de datos Producto\_atributo. }
\label{table:dic-ProductoAtributo}
\end{table}
\FloatBarrier

