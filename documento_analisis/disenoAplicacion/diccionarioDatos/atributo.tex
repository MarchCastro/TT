\title{\textbf{
Tabla de diccionario de datos Atributo
}} \\

El cuadro \ref{table:dic-Atributo} indica la descripción correspondiente de la entidad Atributo.
\label{Entidad-Atributo}
\FloatBarrier
\begin{table}[htb]
\setlength\extrarowheight{2pt}
\begin{tabular}{|p{2.5cm}|p{1.5cm}|p{1.5cm}|p{1.5cm}|p{1.5cm}|p{5.5cm}|}
	\hline
	\multicolumn{2}{|c|}
	{{
		\cellcolor[HTML]{00009B}{\color[HTML]{FFFFFF} Nombre de la relación: }		
	}} &
	\multicolumn{4}{|c|} {{  Atributo }} \\
	\hline
	\multicolumn{2}{|c|} 
	{{
		\cellcolor[HTML]{00009B}{\color[HTML]{FFFFFF} Objetivo: }		
	}} &
	\multicolumn{4}{|p{10cm}|} {{ Tabla que almacenará los valores de los tipos de atributos que puede tener un producto. }} \\
	\hline
	\rowcolor[HTML]{3166FF}
	{\color[HTML]{FFFFFF} Campo }  & 
	{\color[HTML]{FFFFFF} Tipo de dato } & 
	{\color[HTML]{FFFFFF} Tamaño } & 
	{\color[HTML]{FFFFFF} Constraint } & 
	{\color[HTML]{FFFFFF} No nulo } & 
	{\color[HTML]{FFFFFF} Descripción } \\ 
	\hline
	\cellcolor[HTML]{9B9B9B}{\color[HTML]{FFFFFF} id\_atributo } &
	integer &
	10 &
	PK &
	X  & 
	Identificador del atributo. \\
	\hline
	\cellcolor[HTML]{9B9B9B}{\color[HTML]{FFFFFF} fk\_tipo\_atributo } &
	integer &
	10 &
	FK &
	X  & 
	Identificador del tipo de atributo. \\
	\hline
	\cellcolor[HTML]{9B9B9B}{\color[HTML]{FFFFFF} valor } &
	varchar &
	100 &
	 &
	X  & 
	Valor específico del tipo de atributo. \\
	\hline
\end{tabular}
\caption{Tabla de diccionario de datos Atributo. }
\label{table:dic-Atributo}
\end{table}
\FloatBarrier



