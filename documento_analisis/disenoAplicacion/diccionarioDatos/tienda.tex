\title{\textbf{
Tabla de diccionario de datos Tienda
}}\\

El cuadro \ref{table:dic-Tienda} indica la descripción correspondiente de la entidad Tienda.
\label{Entidad-Tienda}
\FloatBarrier
\begin{table}[htb]
\setlength\extrarowheight{2pt}
\begin{tabular}{|p{2.5cm}|p{1.5cm}|p{1.5cm}|p{1.5cm}|p{1.5cm}|p{5.5cm}|}
	\hline
	\multicolumn{2}{|c|}
	{{
		\cellcolor[HTML]{00009B}{\color[HTML]{FFFFFF} Nombre de la relación: }		
	}} &
	\multicolumn{4}{|c|} {{ Tienda }} \\
	\hline
	\multicolumn{2}{|c|} 
	{{
		\cellcolor[HTML]{00009B}{\color[HTML]{FFFFFF} Objetivo: }		
	}} &
	\multicolumn{4}{|c|} {{ Tabla que almacenará los datos básicos de una tienda departamental. }} \\
	\hline
	\rowcolor[HTML]{3166FF} 
	{\color[HTML]{FFFFFF} Campo }  & 
	{\color[HTML]{FFFFFF} Tipo de dato } & 
	{\color[HTML]{FFFFFF} Tamaño } & 
	{\color[HTML]{FFFFFF} Constraint } & 
	{\color[HTML]{FFFFFF} No nulo } & 
	{\color[HTML]{FFFFFF} Descripción } \\ 
	\hline
	\cellcolor[HTML]{9B9B9B}{\color[HTML]{FFFFFF} id\_tienda } &
	bigint &
	20 &
	PK &
	X  & 
	Identificador de la tienda. \\
	\hline
	\cellcolor[HTML]{9B9B9B}{\color[HTML]{FFFFFF} fk\_id\_categoria } &
	bigint &
	20 &
	FK  &
	X  & 
	Identificador de la categoría de la tienda. \\
	\hline
	\cellcolor[HTML]{9B9B9B}{\color[HTML]{FFFFFF} nombre } &
	varchar &
	150 &
	 &
	X  & 
	Nombre de la tienda.   \\ 
	\hline	
	\cellcolor[HTML]{9B9B9B}{\color[HTML]{FFFFFF} direccion } &
	text &
	 &
	 &
	 & 
	Dirección de la tienda.   \\ 
	\hline
	\cellcolor[HTML]{9B9B9B}{\color[HTML]{FFFFFF} posicion\_geografica } &
	point &
	 &
	 &
	X  & 
	Coordenada geográfica en la que se encuentra la tienda.   \\ 
	\hline	
\end{tabular}
\caption{Tabla de diccionario de datos Tienda. }
\label{table:dic-Tienda}
\end{table}
\FloatBarrier

