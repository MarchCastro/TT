\title{\textbf{
Tabla de diccionario de datos Método\_pago
}}\\

El cuadro \ref{table:dic-MetodoPago} indica la descripción correspondiente de la entidad Método\_pago.
\label{Entidad-Metodo_pago}
\FloatBarrier
\begin{table}[htb]
\setlength\extrarowheight{2pt}
\begin{tabular}{|p{2.5cm}|p{1.5cm}|p{1.5cm}|p{1.5cm}|p{1.5cm}|p{5.5cm}|}
	\hline
	\multicolumn{2}{|c|}
	{{
		\cellcolor[HTML]{00009B}{\color[HTML]{FFFFFF} Nombre de la relación: }		
	}} &
	\multicolumn{4}{|c|} {{ Método\_pago }} \\
	\hline
	\multicolumn{2}{|c|} 
	{{
		\cellcolor[HTML]{00009B}{\color[HTML]{FFFFFF} Objetivo: }		
	}} &
	\multicolumn{4}{|c|} {{ Tabla que almacenará los métodos de pago. }} \\
	\hline
	\rowcolor[HTML]{3166FF} 
	{\color[HTML]{FFFFFF} Campo }  & 
	{\color[HTML]{FFFFFF} Tipo de dato } & 
	{\color[HTML]{FFFFFF} Tamaño } & 
	{\color[HTML]{FFFFFF} Constraint } & 
	{\color[HTML]{FFFFFF} No nulo } & 
	{\color[HTML]{FFFFFF} Descripción } \\ 
	\hline
	\cellcolor[HTML]{9B9B9B}{\color[HTML]{FFFFFF} id\_metodo\_pago } &
	smallint &
	5 &
	PK &
	X  & 
	Identificador del método de pago. \\
	\hline
	\cellcolor[HTML]{9B9B9B}{\color[HTML]{FFFFFF} nombre } &
	varchar &
	100 &
	 &
	X  & 
	Nombre del método de pago.   \\ 
	\hline	
\end{tabular}
\caption{Tabla de diccionario de datos Método\_pago.  }
\label{table:dic-MetodoPago}
\end{table}
\FloatBarrier

